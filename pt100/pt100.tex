This chapter illustrates the use of the least squares method in finding the
voltage across the PT-100 RTD (Resistance Temperature Detector) as a
function of temperature.

\section{Training Data}
The training data gathered by the PT-100 to train the Arduino is shown in Table
\ref{tab:train}.

\begin{table}[!ht]
    \centering
    \input{pt100/tables/training_data.tex}
    \caption{Training data.}
    \label{tab:train}
\end{table}

The C++ source file
\begin{lstlisting}
pt100/codes/data.cpp
\end{lstlisting}
was used along with \textit{platformio} to drive the Arduino. The effective 
schematic circuit diagram is shown in \autoref{fig:ckt}.

\begin{figure}[!ht]
    \centering
    \begin{circuitikz} \draw
        (0,0) to[battery1, l=$5\ V$, invert] (0,2)
        to[R, l^=$10\ \Omega$] (3,2) to[short, -o] (5,2);
        \draw (3,2) to[R, l^=$P\ \Omega$] (3,0)
        -- (0,0);
        \draw (3,0) to[short, -o] (5,0);
    \end{circuitikz}
    \caption{Schematic Circuit Diagram to Measure the Output of PT-100 ($P$).}
    \label{fig:ckt}
\end{figure}

\section{Model}

For the PT-100, we use the Callendar-Van Dusen equation
\begin{align}
    V(T) &= V(0)\brak{1+AT+BT^2} \\
    \implies c &= \vec{n}^\top\vec{x} \label{eq:model}
\end{align}
where
\begin{align}
    c = V(T),\ \vec{n} = V(0)\myvec{1\\A\\B},\ \vec{x} = \myvec{1\\T\\T^2}
    \label{eq:x-y-theta-def}
\end{align}

For multiple points, \eqref{eq:model} becomes
\begin{align}
    \vec{X}^\top\vec{n} = \vec{C}
    \label{eq:lsq-eqn}
\end{align}
where
\begin{align}
    \vec{X} &= \myvec{1&1&\ldots&1\\T_1&T_2&\ldots&T_n\\T_1^2&T_2^2&\ldots&T_n^2} \\
    \vec{C} &= \myvec{V\brak{T_1}\\V\brak{T_2}\\\vdots\\V\brak{T_n}}
\end{align}
and $\vec{n}$ is the unknown.

\section{Solution}
We approximate $\vec{n}$ by using the least sqaures method. The Python code 
\texttt{codes/lsq.py} solves for $\vec{n}$.

The calculated value of $\vec{n}$ is
\begin{align}
    \vec{n} = \myvec{1.6547\\3.199\times10^{-3}\\-3.9599\times10^{-6}}
    \label{eq:opt-theta}
\end{align}

The approximation is shown in Fig. \ref{fig:train}.
\begin{figure}[!ht]
    \centering
    \includegraphics[width=\columnwidth]{pt100/figs/train.png}
    \caption{Training the model.}
    \label{fig:train}
\end{figure}

Thus, the approximate model is given by
\begin{align}
    V(T) &= 1.6547 + \brak{3.199\times10^{-3}}T \nonumber \\
         &- \brak{3.9599\times10^{-6}}T^2
    \label{eq:opt-model}
\end{align}

Notice in \eqref{eq:opt-model} that the coefficient of $T^2$ is negative, and 
hence the governing function is strictly concave. Hence, we cannot use gradient 
descent methods to solve this problem.

\section{Validation}
The validation dataset is shown in Table \ref{tab:valid}. The results of the 
validation are shown in Fig. \ref{fig:valid}.
\begin{table}[!ht]
    \centering
    \input{pt100/tables/validation_data.tex}
    \caption{Validation data.}
    \label{tab:valid}
\end{table}
\begin{figure}[!ht]
    \centering
    \includegraphics[width=\columnwidth]{pt100/figs/valid.png}
    \caption{Validating the model.}
    \label{fig:valid}
\end{figure}
