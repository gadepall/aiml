
%\renewcommand{\theequation}{\theenumi}
%\begin{enumerate}[label=\arabic*.,ref=\theenumi]
\begin{enumerate}[label=\thechapter.\arabic*.,ref=\thechapter.\theenumi]
%\begin{enumerate}
%\numberwithin{equation}{enumi}
%\item The area of a triangle with vertices $\vec{A}, \vec{B}, \vec{C}$ is given by 
%\begin{align}
%  \label{eq:area3d}
% \frac{1}{2} \norm{\brak{\vec{A} - \vec{B}} \times \brak{\vec{A} - \vec{C}}}
%\end{align}
\item The lines 
    \begin{align}
        \vec{x} = \vec{x_1} + \lambda_1\vec{m_1} \label{eq:12/11/2/16/L1-gen} \\
        \vec{x} = \vec{x_2} + \lambda_2\vec{m_2} \label{eq:12/11/2/16/L2-gen}
    \end{align}
    intersect if
    \begin{align}
    %    \lambda_1\vec{m_1} - \lambda_2\vec{m_2} &= \vec{x_2} - \vec{x_1} \\
       \vec{M}\bm{\lambda} &= \vec{x_2} - \vec{x_1}
        \label{eq:12/11/2/16/intersect-cond}
    \end{align}
    where
    \begin{align}
        \vec{M} \triangleq \myvec{\vec{m_1} & \vec{m_2}} \label{eq:12/11/2/16/M-def} \\
        \bm{\lambda} \triangleq \myvec{\lambda_1\\-\lambda_2}
        \label{eq:12/11/2/16/lambda-def}
    \end{align}
\item 
	The closest points on two skew lines are given by 
    \begin{align}
	    \vec{M}^\top \vec{M}\bm{\lambda} = \vec{M}^\top\brak{\vec{x_2}-\vec{x_1}}
        \label{eq:12/11/2/16/lambda-eqn}
    \end{align}
	\solution
    For the lines defined in \eqref{eq:12/11/2/16/L1-gen} and \eqref{eq:12/11/2/16/L2-gen},
Suppose the closest points on both lines are
    \begin{align}
        \vec{A} = \vec{x_1} + \lambda_1\vec{m_1} \label{eq:12/11/2/16/a-def} \\
        \vec{B} = \vec{x_2} + \lambda_2\vec{m_2}
        \label{eq:12/11/2/16/b-def}
    \end{align}
    Then, $AB$ is perpendicular to both lines, hence
    \begin{align}
        \vec{m_1}^\top\brak{\vec{A}-\vec{B}} = 0 \\
        \vec{m_2}^\top\brak{\vec{A}-\vec{B}} = 0 \\
        \implies \vec{M}^\top\brak{\vec{A}-\vec{B}} = \vec{O}
        \label{eq:12/11/2/16/perp-vec}
    \end{align}
    Using \eqref{eq:12/11/2/16/a-def} and \eqref{eq:12/11/2/16/b-def} in \eqref{eq:12/11/2/16/perp-vec},
    \begin{align}
        \vec{M}^\top\brak{\vec{x_1}-\vec{x_2} + \vec{M}\bm{\lambda}} = \vec{0} \\
    \end{align}
    yielding
        \ref{eq:12/11/2/16/lambda-eqn}.
%\renewcommand{\theequation}{\theenumi}
%%\begin{enumerate}[label=\arabic*.,ref=\theenumi]
%\begin{enumerate}[label=\thesubsection.\arabic*.,ref=\thesubsection.\theenumi]
%\numberwithin{equation}{enumi}
%
\end{enumerate}
